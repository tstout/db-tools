%
%  -- db-tools design rationale --
%
% To convert to markdown:
% pandoc -f latex -t markdown readme.tex
%
\documentclass[10pt]{article}
\usepackage{url}
\usepackage{pgfplots}
\usepackage{hyperref}
\usepackage{moreverb}           % contains verbatimtab
\usepackage{textcomp}
\title{db-tools}
\date{October 2013}
\begin{document}
 \maketitle

\section*{Design Thoughts}

I have done some prototyping of DB IO using Empire DB. At
first I liked it. After some reflection however, I decided it is not a 
dependency I want to bring along. Some features are nice, but I
realized that 
type safe
queries are not really a major benefit for me. I'm leaning towards raw SQL in
resource files...dynamic SQL is a strength, not a weakness that should
not be abstracted away into a static object model.
Furthermore, SQL is very powerful when used for it's intended purpose: querying
data. I don't want to limit the expressiveness of a query to an OO
wraper. I think SQL breaks down when you try to use it to do complex,
procedural logic. It is great at maniuplating sets of data. Use it for
this only, and it becomes a powerful tool, as intended. 
\\
\\
Many schema versioning tools exist. ActiveRecord migrations and
liquibase come to mind.
%
% More here on why I don't just simply use one of those.
%

\section*{Major Components}
\begin{itemize}
  \item ANTLR Grammar defining the custom DDL language
  \item Symbol table and other related objects to process the metadata
   harvested from the generated ANTLR parser.
   \item Schema Mutator - apply changes to a DB schema
 \item Code generation from parsed metadata (more on this later)
 
 \end{itemize}

\section*{Minor Components}
\begin {itemize}
\item WIP
\end{itemize}

\section*{Code Generation or Runtime Interpreter?}
A primary motivator for creating an external DSL is the desire to have
access to the DB metadata without requiring a database instance to be
actually running. I'm familiar with the typical technique of
generating source code and have a build step to generate the desired
functionality. In the past I have thought that requiring an additional
step in the build process to invoke a custom compiler tool was a
burden. With tools like gradle and rake, I'm not so sure now.

\subsection*{Implementation options}
\begin{itemize}
  \item Clojure's genclass (and maybe protocols for monkey-patching)
\end{itemize}

   



%Created by \LaTeX\
\end{document}
